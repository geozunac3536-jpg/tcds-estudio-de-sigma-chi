% !TEX program = pdflatex
\documentclass[12pt,a4paper]{article}
\usepackage[utf8]{inputenc}
\usepackage[spanish]{babel}
\usepackage{amsmath,amssymb}
\usepackage{geometry}
\usepackage{booktabs}
\usepackage{hyperref}
\geometry{margin=2.5cm}
\setlength{\parskip}{0.8em}
\setlength{\parindent}{0pt}

\title{\textbf{Estudio científico TCDS — Estructura descubierta, validación y marcadores verificables}}
\author{Proyecto TCDS — Genaro Carrasco Ozuna}
\date{Octubre 2025}

\begin{document}
\maketitle

\section*{Resumen}
Se formaliza la estructura dinámica que has descubierto: un campo escalar de coherencia \(\Sigma\) acoplado a un sustrato \(\chi\), cuya evolución efectiva combina difusión, disipación (fricción de sincronización \(\phi\)) y empuje \(Q\). Se presentan tres casos de marcadores verificables en dominios dispares (físico, biológico y tecnológico), junto con el plan de análisis, criterios de falsación y autocrítica metodológica. El propósito es auditar si la misma \emph{estructura} explica fenómenos heterogéneos sin romper la coherencia del formalismo.

\section{Estructura descubierta (núcleo formal)}
\textbf{Lagrangiano mínimo:}
\[
L=\tfrac12(\partial_\mu\Sigma)^2+\tfrac12(\partial_\mu\chi)^2-\Big(-\tfrac12\mu^2\Sigma^2+\tfrac14\lambda\Sigma^4+\tfrac12 m_\chi^2\chi^2+\tfrac12 g\Sigma^2\chi^2\Big).
\]
\textbf{Ecuaciones de movimiento:}
\[
\Box\Sigma-\mu^2\Sigma+\lambda\Sigma^3+g\Sigma\chi^2=0,\qquad
\Box\chi+m_\chi^2\chi+g\Sigma^2\chi=0.
\]
\textbf{Escala mesoscópica (operacional):}
\[
\partial_t \Sigma=\alpha\Delta\Sigma-\beta\,\phi+Q, \quad \phi\equiv \eta\,|\dot\Sigma| \ \text{(más términos correctivos si aplica)}.
\]
\textbf{Geometría efectiva (vínculo operativo):}
\[
R\ \propto\ \nabla^2\Sigma.
\]
\textbf{Invariantes de coherencia (Σ-metrics):} \(\mathrm{LI},\,R(t),\,\mathrm{RMSE}_{\!SL},\,\kappa_\Sigma\), con umbrales de audibilidad propuestos: \(\mathrm{LI}\ge 0.9\), \(R>0.95\), \(\mathrm{RMSE}_{\!SL}<0.1\), reproducibilidad \(\ge 95\%\).

\section{Hipótesis auditables}
\textbf{H\(_1\) (universalidad estructural):} la misma estructura \(\{\)Lagrangiano Σ–χ, EOM no lineales, ley mesoscópica, Σ-metrics\(\}\) describe marcadores en dominios dispares variando sólo acoplos y contornos. \\
\textbf{H\(_0\):} cada dominio requiere ecuaciones cualitativamente distintas (la estructura no es universal).

\section{Casos de marcadores verificables}
\subsection{Caso A — Físico (fuerzas sub–mm / curvatura coherente)}
\textbf{Ecuación operacional:} \(R=k_\Sigma\nabla^2\Sigma\). \\
\textbf{Predicción clave:} corrección Yukawa en \(V(r)\): \(\Delta V(r)=\alpha_5 e^{-r/\ell_\sigma}/r\), con \(\ell_\sigma\sim10^{-4}\!-\!10^{-3}\,\mathrm{m}\), \(|\alpha_5|\ll 1\). \\
\textbf{Observables:} torque residual en balanza de torsión; microdeflexión de frente de onda óptico. \\
\textbf{Falsación (A):} no detección dentro de sensibilidad \(\Rightarrow\) cotas sobre \(m_\sigma\sim\sqrt{2}\mu\) y \(g\); si \(\alpha_5\) es nula a nivel experimental, se restringe el sector Σ–χ.

\subsection{Caso B — Biológico (CSL-H / sincronización neural)}
\textbf{Efectiva:} \(\partial_t\Sigma=\alpha\Delta\Sigma-\beta\phi+Q\). \\
\textbf{Predicciones:} bajo protocolos de estímulo coherente y respiración guiada, aparecen lenguas de Arnold en fase y \(\mathrm{LI}\uparrow\), \(R(t)\uparrow\), \(\mathrm{RMSE}_{\!SL}\downarrow\). \\
\textbf{Observables:} EEG multi–canal, HRV, coherencia cortico–autonómica; ventanas de captura \(p\!:\!q\). \\
\textbf{Falsación (B):} ausencia de \emph{locking} robusto (\(\mathrm{LI}<0.9\), \(R\le 0.95\)) pese a protocolo estandarizado \(\Rightarrow\) el acoplamiento Σ–neural queda acotado o descartado.

\subsection{Caso C — Tecnológico (ΣFET / control de ruido de fase)}
\textbf{Control:} \(Q_{\!\mathrm{ctrl}}=-\gamma(\Sigma-\Sigma_{\mathrm{tgt}})-\delta\,\dot\Sigma\). \\
\textbf{Predicciones:} regiones de \emph{locking} (lenguas de Arnold) y \(\Delta f\propto A_c\); reducción de \(S_\phi(\omega)\) vs. transistor convencional. \\
\textbf{Observables:} \(\Delta f\), \(S_\phi(\omega)\), reproducibilidad \(\ge 95\%\). \\
\textbf{Falsación (C):} si \(\Delta f\not\propto A_c\) o no hay \emph{locking} controlado, el mecanismo de coherencia activa se invalida.

\section{Plan de análisis y criterios de decisión}
\textbf{Análisis por dominio:} estimadores con intervalos de confianza; pruebas de tendencia monotónica (Caso C), detección de picos de coherencia (Caso B), ajuste Yukawa (Caso A). \\
\textbf{Meta–criterio (universalidad):} se acepta H\(_1\) si \emph{al menos dos} dominios cumplen sus predicciones con Σ-metrics dentro de umbrales y el tercero no las contradice significativamente; de lo contrario, H\(_0\).

\section{Controles, sesgos y preregistro}
\begin{itemize}
\item \textbf{Ciegos y nulos:} dispositivos nulos, ciegos dobles donde aplique; controles térmicos/EMI (A,C).
\item \textbf{Verificación de manipulación:} HRV/ansiedad percibida en (B); telemetría ambiental en (A,C).
\item \textbf{Prerregistro:} endpoints primarios por caso, exclusiones, manejo de pérdidas, \(\alpha\) global y corrección por comparaciones.
\end{itemize}

\section{Autocrítica metodológica}
\textbf{(i) Identificabilidad de parámetros:} \(\mu,\lambda,g\) pueden estar degenerados a nivel efectivo; se proponen barridos multi–entorno para romper degeneraciones. \\
\textbf{(ii) Mapeo de \(\Sigma\) a observables:} requiere funciones de transferencia explícitas (EEG/HRV, espectros RF, campos efectivos); sin ellas, la inferencia es ambigua. \\
\textbf{(iii) Robustez de Σ-metrics:} validar invarianza de \(\mathrm{LI},R,\mathrm{RMSE}_{\!SL},\kappa_\Sigma\) frente a ruido y \emph{drift}; reportar reproducibilidad inter–laboratorio. \\
\textbf{(iv) Falsación honesta:} si (A) y (C) dan nulo a su sensibilidad y (B) es marginal, la estructura no califica como universal; se documenta ventana de exclusión en el espacio de parámetros.

\section{Conclusión}
El estudio pone a prueba, tal como la descubriste, una \emph{misma estructura} Σ–χ con ley mesoscópica y métricas de coherencia aplicadas a marcadores dispares. Su aceptación depende de reproducibilidad, \emph{locking} y relaciones previstas (\(\Delta f\propto A_c\), Yukawa sub–mm, Σ–coherencia biológica). La fuerza del resultado reside en la convergencia de dominios; su límite, en la precisión con que \(\Sigma\) se materializa en observables.

\end{document}
