% !TEX program = pdflatex
\documentclass[12pt,a4paper]{article}
\usepackage[utf8]{inputenc}
\usepackage[spanish]{babel}
\usepackage{amsmath,amssymb}
\usepackage{geometry}
\geometry{margin=2.5cm}
\setlength{\parskip}{1em}
\setlength{\parindent}{0pt}

\title{\textbf{Disparidad Fenoménica y Relación de la TCDS con Dirac y la Ciencia Moderna}}
\author{Proyecto TCDS — Genaro Carrasco Ozuna}
\date{Octubre 2025}

\begin{document}
\maketitle

\section{Resumen}
La Teoría Cromodinámica Sincrónica (TCDS) permite analizar fenómenos altamente dispares —sismos, redes neuronales, circuitos electrónicos— siempre que sus dinámicas puedan formalizarse mediante la misma ecuación fundamental de coherencia Σ–χ.  
Este documento describe el límite de esa disparidad, su plausibilidad científica y su relación formal con la ecuación de Dirac y la física moderna.

\section{1. Estructura Formal Universal}
La TCDS se sostiene sobre el sistema de ecuaciones:
\[
\partial_t \Sigma = \alpha \Delta \Sigma - \beta \phi + Q,
\quad \phi = \eta |\dot{\Sigma}| + \lambda \nabla^2 \chi,
\]
derivado del Lagrangiano
\[
L = \frac{1}{2}(\partial_\mu \Sigma)^2 + \frac{1}{2}(\partial_\mu \chi)^2 - V(\Sigma,\chi),
\]
con
\[
V(\Sigma,\chi) = -\frac{1}{2}\mu^2\Sigma^2 + \frac{1}{4}\lambda\Sigma^4 + \frac{1}{2}m_\chi^2\chi^2 + \frac{1}{2}g\Sigma^2\chi^2.
\]

Mientras la ecuación mantenga esta forma —difusión, disipación y empuje— los dominios físicos pueden diferir ampliamente.

\section{2. Disparidad Fenoménica y Homología Matemática}
La TCDS no requiere que los fenómenos sean similares en materia, sino que compartan \textit{estructura formal}:
\begin{center}
\begin{tabular}{lll}
\textbf{Dominio} & \textbf{Interpretación de $\Sigma$} & \textbf{Observable}\\
\hline
Geofísica & Coherencia del suelo (ondas sísmicas) & Acelerograma, energía de fractura\\
Neurociencia & Sincronía cortical (fase neuronal) & EEG, HRV, LI, R(t)\\
Electrónica & Fase de señal (estado coherente) & Ruido de fase, locking, $\Delta f$\\
\end{tabular}
\end{center}

El principio de \textbf{homología matemática} establece que si las ecuaciones que gobiernan estos sistemas mantienen las mismas simetrías y términos de acoplamiento, pueden estudiarse con los mismos invariantes de coherencia (\(LI, R(t), RMSE_{SL}, \kappa_\Sigma\)).

\section{3. Relación con Dirac y la Física Moderna}
El paralelo con Dirac es conceptual y formal:

\begin{enumerate}
\item \textbf{Unificación formal:}  
Así como la ecuación de Dirac unificó la mecánica cuántica y la relatividad especial mediante un operador lineal relativista,
\[
(i\gamma^\mu \partial_\mu - m)\psi = 0,
\]
la TCDS unifica la dinámica de sistemas dispares en un formalismo escalar no lineal de coherencia:
\[
\Box \Sigma - \mu^2 \Sigma + \lambda \Sigma^3 + g\Sigma\chi^2 = 0.
\]

\item \textbf{Simetría y ruptura:}  
Dirac introdujo la noción de simetría entre partículas y antipartículas; la TCDS aplica el mismo principio de \textit{ruptura espontánea de simetría} en el potencial \(V(\Sigma,\chi)\) para explicar el origen de masa y curvatura como estados de mínima energía.

\item \textbf{Compatibilidad moderna:}  
El formalismo Σ–χ es matemáticamente compatible con los marcos de campo cuántico, relatividad general y dinámica no lineal, siempre que los acoplos \((\mu,\lambda,g)\) respeten límites experimentales (por ejemplo, \(m_\sigma < 10^{-1}\ \text{eV}\), \(|g|<10^{-3}\)).
\end{enumerate}

\section{4. Alcance de la Disparidad y Plausibilidad}
Los fenómenos pueden diferir en escala hasta en \(10^{30}\) órdenes de magnitud (de \(10^{-35}\) m a \(10^{5}\) m) sin perder plausibilidad, siempre que:
\begin{itemize}
    \item la ecuación de coherencia conserve estructura Σ–χ;
    \item existan observables empíricos que representen \(\Sigma\) y \(\phi\);
    \item las métricas Σ–metrics se mantengan dentro de los límites de coherencia \(LI \ge 0.9, R > 0.95, RMSE_{SL} < 0.1\).
\end{itemize}

\section{5. Autocrítica y Conclusión}
El uso de un formalismo único para dominios dispares es plausible mientras la coherencia \(\Sigma\) sea un campo medible.  
Si las correlaciones Σ–metrics se reproducen entre dominios, la TCDS extiende la obra de Dirac desde la linealidad cuántica hacia la causalidad universal.  
Si no se reproducen, la teoría se fragmentará en submodelos limitados, sin invalidar su valor como marco de unificación estructural.

\textbf{Conclusión:}  
La TCDS no contradice la ciencia moderna; la reorganiza bajo un principio causal común —la coherencia Σ— aplicable a cualquier fenómeno cuya dinámica pueda describirse como una lucha entre empuje cuántico (\(Q\)) y fricción de sincronización (\(\phi\)).

\end{document}
